\documentclass{article}

\usepackage{amsmath}
\usepackage{amssymb}
\usepackage{amsthm}
\usepackage{fullpage}
\usepackage{hyperref}

\newcommand{\dunw}{d_{\mathrm{unw}}}
\newcommand{\eps}{\varepsilon}

\newtheorem{definition}{Definition}
\newtheorem{theorem}{Theorem}

\title{On $\eps$-Nets, Distance Oracles, and Metric Embeddings}
\author{Ilya Razenshteyn}
\date{}

\begin{document}
    \maketitle
    \section{Introduction}

    Let $G$ be a weighted undirected graph with unique shortest paths.
    In~\cite{ADFGW11} the following observation is heavily used: the set of all shortest paths of $G$ (by shortest path we mean its set of vertices)
    has VC-dimension~\cite{VC71} at most two.
    We give two new applications of this fact.

    \subsection{Distance oracles}

    The first application is about distance oracles.
    Let $G = (V, E)$ be an undirected unweighted graph with $n$ vertices.
    For vertices $v_1, v_2 \in V$ let $d(v_1, v_2)$ be the distance between $v_1$ and $v_2$.
    Let $\eps, \delta$ be two fixed positive constants.
    In Section~\ref{data_structure}
    we build a data structure of size $O(n^{1 + \delta})$ that given two vertices $v_1, v_2 \in V$ in constant time reports the following:
    \begin{itemize}
        \item If $d(v_1, v_2) < \eps n$, then the data structure reports ``$\perp$'';
        \item If $d(v_1, v_2) \geq \eps n$, then the data structure reports $d(v_1, v_2)$ exactly.
    \end{itemize}

    \subsection{Metric embeddings}

    The second application is about metric embeddings.
    Let $G = (V, E, w)$ be an undirected weighted graph with $n$ vertices.
    For vertices $v_1, v_2 \in V$ let $d(v_1, v_2)$ be the distance between $v_1$ and $v_2$ with respect to $w$,
    and $\dunw(v_1, v_2)$ be the distance with respect to unit weights.

    Suppose we want to approximate metric $d$ with $\ell_1$-norm. We are looking for mappings from $V$ to $\ell_1$ with small distortion. 
    \begin{definition}
        Let us say that a mapping $f \colon V \to (\mathbb{R}^k, \ell_1)$ has distortion $D$ if for every $v_1, v_2 \in V$
        $$
            \frac{d(v_1, v_2)}{D} \leq \|f(v_1) - f(v_2)\|_1 \leq d(v_1, v_2).
        $$
    \end{definition}

    There are many results about $\ell_1$-embeddability. Let us state several of them.

    \begin{definition}
        Doubling dimension of $d$ is the minimum integer $k$ such that every subset $V' \subseteq V$ of diameter $\Delta$
        can be covered with $2^k$ subsets of diameter $\Delta / 2$.
    \end{definition}

    \begin{definition}
        Let us say that $d$ is of negative type if $\sqrt{d}$ is isometrically embeddable into $\ell_2$.
    \end{definition}

    \begin{theorem}
        \label{l1_embeddings}
        There are the following upper-bounds on $D$:
        \begin{itemize}
            \item \cite{B85} For any $d$ one can take $D = O(\log n)$;
            \item \cite{R99, IS07} If $G$ is of fixed genus, then $D = O(\sqrt{\log n})$. 
            \item \cite{GKL03} If $d$ has bounded doubling dimension, then $D = O(\sqrt{\log n})$.
            \item \cite{ALN05} If $d$ is of negative type, then $D = O(\sqrt{\log n} \log \log n)$. 
        \end{itemize}
    \end{theorem}

    But what if we are interested in preserving $d(v_1, v_2)$ only if $\dunw(v_1, v_2) \geq \eps n$?
    In Section~\ref{metric_embeddings} we state and prove a generic black box transformation
    that allows us to replace all occurences of $n$ in Theorem~\ref{l1_embeddings} with $1 / \eps$ for this case.

    In~\cite{ABCDGKNS05} somewhat stronger result about arbitrary and bounded doubling dimension metrics is proved, but since our transformation
    is black box, it automatically holds for fixed genus and negative type metrics.
    \section{VC-dimension and $\eps$-nets}

    \section{Distance oracles}
    \label{data_structure}

    \section{Metric embeddings}
    \label{metric_embeddings}

    \bibliographystyle{alpha}
    \bibliography{../bibtex/ir}
\end{document}
