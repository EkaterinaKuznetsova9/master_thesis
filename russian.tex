\documentclass[12pt]{article}

\usepackage{amsmath}
\usepackage{amssymb}
\usepackage{amsthm}
\usepackage[english,russian]{babel}
\usepackage{fullpage}
\usepackage[utf8]{inputenc}
\usepackage{hyperref}
\usepackage{ccfonts}
\usepackage[T1]{fontenc}

\renewcommand{\bfdefault}{sbc}

\newcommand{\eps}{\varepsilon}
\newcommand{\Rc}{\mathcal{R}}
\newcommand{\dunw}{d_{\mathrm{unw}}}
\newcommand{\set}[1]{\left\{#1\right\}}
\newcommand{\abs}[1]{\left|#1\right|}
\newcommand{\zo}{\set{0, 1}}
\newcommand{\Oc}{\mathcal{O}}
\newcommand{\Pc}{\mathcal{P}}
\newcommand{\Exp}[2]{\mathbf{E}_{#1}\left[#2\right]}
\newcommand{\setst}[2]{\set{#1 \mid #2}}

\newtheorem{definition}{Определение}
\newtheorem{problem}{Задача}
\newtheorem{theorem}{Теорема}

\title{Покрытия длинных кратчайших путей в ненаправленных графах и системы пересадок в булевом кубе}
\author{Илья Разенштейн}

\begin{document}
    \maketitle
    \section{Введение}
    \subsection{Постановка задач}
    Пусть $G = (V, E, w)$~--- взвешенный ненаправленный граф с неотрицательными весами на ребрах.
    Граф $G$ порождает две метрики на вершинах: обозначим $d(v_1, v_2)$ кратчайшее расстояние между вершинами $v_1$
    и $v_2$, а $\dunw(v_1, v_2)$~--- кратчайшее расстояние в невзвешенной версии $G$.

    Поставим две задачи.
    \begin{problem}[Задача об $\eps$-покрытиях]
        Пусть $\eps > 0$~--- положительный параметр. Назовем множество вершин $N \subseteq V$ \emph{$\eps$-покрытием},
        если для любых вершин $v_1, v_2 \in V$ таких, что $\dunw(v_1, v_2) \geq \eps n$, найдется вершина $u \in N$
        такая, что $d(v_1, v_2) = d(v_1, u) + d(u, v_2)$.

        Каков минимальный размер $\eps$-покрытия графа $G$?
    \end{problem}
    \begin{problem}[Задача о системах пересадок]
        Назовем отображение $L \colon V \to 2^V$ \emph{системой пересадок}, если для любых вершин $v_1, v_2 \in V$
        найдется вершина $u \in L(v_1) \cap L(v_2)$ такая, что $d(v_1, v_2) = d(v_1, u) + d(u, v_2)$.
        \emph{Размером} системы пересадок $L$ назовем величину $\sum_{v \in V} \abs{L(v)}$. 

        Каков минимальный размер системы пересадок графа $G$?
    \end{problem}
    \subsection{Задача об $\eps$-покрытиях}
    В разделе~\ref{section_eps_covers} мы изучаем задачу об $\eps$-покрытиях.
    В разделе~\ref{subsection_general_graphs} с помощью теории размерности Вапника-Червоненкиса~\cite{VC71}
    мы доказываем существование $\eps$-покрытий размера $O(\log(1 / \eps) / \eps)$.
    Мы используем наблюдение из~\cite{ADFGW11}: система уникальных кратчайших путей имеет маленькую размерность
    Вапника-Червоненкиса.
    Также мы доказываем сверхлинейную нижнюю оценку $\omega(1 / \eps)$ на размер $\eps$-покрытия.
    Для этого мы используем плотностную версию теоремы Хейлса-Джуетта~\cite{P09} из аддитивной комбинаторики.
    Отметим, что в статье~\cite{A10} применяется такая же техника для похожей комбинаторной задачи.
    Далее в разделе~\ref{subsection_bounded_treewidth} мы доказываем линейную верхнюю оценку $O(1 / \eps)$
    на размер $\eps$-покрытия в графах ограниченной древесной ширины.

    Далее в разделе~\ref{subsection_applications} мы описываем два приложения маленьких $\eps$-покрытий.
    \subsubsection{Оракул больших расстояний}
    Пусть $G = (V, E)$~--- ненаправленный \emph{невзвешенный} граф.
    Мы хотим построить структуру данных, которая по паре вершин $v_1, v_2 \in V$ сообщает одно из двух:
    если $d(v_1, v_2) < \eps n$, то структура должна сообщить об ошибке, если же $d(v_1, v_2) \geq \eps n$,
    то структура должна сообщить точное значение $d(v_1, v_2)$.

    В разделе~\ref{subsubsection_distance_oracle} мы строим для любых $\eps, \delta > 0$
    такую структуру данных размера $O(n^{1 + \delta})$, которая умеет отвечать на запрос за константное время.
    Для этого мы комбинируем факт наличия маленьких $\eps$-покрытий и оракул Торупа-Цвика~\cite{TZ05}.
    \subsubsection{Вложение метрик в $\ell_1$}
    Пусть $G = (V, E, w)$~--- взвешенный ненаправленный граф.
    Будем интересоваться, насколько хорошо можно приблизить метрику $d$ $\ell_1$-нормой.
    Следующая классическая теорема дает оптимальную оценку.
    \begin{theorem}[\cite{B85}]
        \label{theorem_bourgain}
        Существует отображение $f \colon V \to \ell_1^{O(\log^2 n)}$ такое, что для любых вершин 
        $v_1, v_2 \in V$ выполнены следующие неравенства:
        $$
            \frac{d(v_1, v_2)}{O(\log n)} \leq \|f(v_1) - f(v_2)\|_1 \leq d(v_1, v_2).
        $$
    \end{theorem}

    Но что если мы интересуемся только сохранением расстояний между вершинами $v_1, v_2 \in V$ такими, что
    $\dunw(v_1, v_2) \geq \eps n$? В разделе~\ref{subsubsection_metric_embeddings} мы докажем такую теорему.

    \begin{theorem}
        Существует отображение $f \colon V \to \ell_1^{O(\log n + \log^2 (1 / \eps))}$ такое, что для любых вершин 
        $v_1, v_2 \in V$ выполнено
        $$
            \frac{d(v_1, v_2)}{O(\log (1/\eps))} \leq \|f(v_1) - f(v_2)\|_1,
        $$
        и если $\dunw(v_1, v_2) \geq \eps n$, то дополнительно выполнено
        $$
            \|f(v_1) - f(v_2)\|_1 \leq d(v_1, v_2).
        $$
    \end{theorem}
    В работе~\cite{ABCDGKNS05} доказана более сильная версия этой теоремы, но наше доказательство непосредственно
    обобщается на
    специальные классы метрик, для которых теорему~\ref{theorem_bourgain} можно усилить (например, на метрики
    отрицательного типа~\cite{ALN07}).
    \subsection{Задача о системах пересадок}

    Задача о системах пересадок изучалась как с теоретической, так и с практической точки зрения.
    В работе~\cite{CHKZ02} приводится полиномиальный приближенный алгоритм, который строит $O(\log n)$-приближение
    для систем пересадок минимального размера.
    В работе~\cite{AFGW10} доказывается, что в графах ограниченной шоссейной размерности существуют системы пересадок
    почти линейного размера.
    В работе~\cite{ADGW11} показывается, что системы пересадок можно с успехом использовать для практического
    вычисления расстояний в дорожных сетях континентального размера.

    В разделе~\ref{section_hub_labels} мы получаем практически точную оценку на размер минимальной системы пересадок
    для булева куба $\zo^n$. Мы доказываем, что он равен $2.5^{(1 + o(1)) n}$.
    \section{Задача об $\eps$-покрытиях}
    \label{section_eps_covers}
    \subsection{Графы общего вида}
    \label{subsection_general_graphs}
    \subsubsection{Размерность Вапника-Червоненкиса и $\eps$-сети}
    \subsubsection{Верхняя оценка}
    \subsubsection{Нижняя оценка}
    \subsection{Графы ограниченной древесной ширины}
    \label{subsection_bounded_treewidth}
    \subsection{Приложения}
    \label{subsection_applications}
    \subsubsection{Оракул больших расстояний}
    \label{subsubsection_distance_oracle}
    \subsubsection{Вложение метрик в $\ell_1$}
    \label{subsubsection_metric_embeddings}
    \section{Системы пересадок в булевом кубе}
    \label{section_hub_labels}
    \section{Заключение}
    \bibliographystyle{alpha}
    \bibliography{../bibtex/ir}
\end{document}
