\documentclass{article}

\usepackage{amsmath}
\usepackage{amssymb}
\usepackage{amsthm}
\usepackage[english,russian]{babel}
\usepackage{fullpage}
\usepackage[utf8]{inputenc}

\newcommand{\eps}{\varepsilon}

\title{Длинные кратчайшие пути в ненаправленных графах}

\begin{document}
    \maketitle
    \section{Введение}
    \section{Графы общего вида}
    \subsection{Размерность Вапника-Червоненкиса и $\eps$-сети}
    \subsection{Верхняя оценка}
    \subsection{Нижняя оценка}
    \cite{A10}
    \section{Графы ограниченной древесной ширины}
    \subsection{Древесная ширина}
    \subsection{Случай деревьев}
    \subsection{Верхняя оценка}
    \section{Приложения}
    \subsection{Оракул больших расстояний}
    \subsection{Приближение метрик нормами}
    \subsection{Задача о разреженном разрезе}
    \section{Заключение}
    \bibliographystyle{plain}
    \bibliography{../bibtex/ir}
\end{document}
