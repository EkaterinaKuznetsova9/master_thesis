\documentclass{article}

\usepackage{amsmath}
\usepackage{amssymb}
\usepackage{amsthm}
\usepackage[english,russian]{babel}
\usepackage{fullpage}
\usepackage[utf8]{inputenc}

\newcommand{\eps}{\varepsilon}
\newcommand{\Rc}{\mathcal{R}}

\newtheorem{definition}{Определение}

\title{Длинные кратчайшие пути в ненаправленных графах}

\begin{document}
    \maketitle
    \section{Введение}
    \section{Графы общего вида}
    \subsection{Размерность Вапника-Червоненкиса и $\eps$-сети}

    Пусть $X$ --- конечное множество, $\Rc \subseteq 2^X$ --- система подмножеств $X$.
    Следующие определения были даны в работе~\cite{VC71} для доказательства равномерного варианта закона больших чисел.

    \begin{definition}
        Подмножество $U \subseteq X$ \emph{разбивается} $\Rc$, если для любого $V \subseteq U$ найдется $W \in \Rc$ такое,
        что $W \cap U = V$.
    \end{definition}

    \subsection{Верхняя оценка на размер $\eps$-сетей}
    \subsection{Нижняя оценка на размер $\eps$-сетей}
    \cite{A10}
    \section{Графы ограниченной древесной ширины}
    \subsection{Древесная ширина}
    \subsection{Случай деревьев}
    \subsection{Верхняя оценка на размер $\eps$-сетей}
    \section{Приложения}
    \subsection{Оракул больших расстояний}
    \subsection{Приближение метрик нормами}
    \subsection{Задача о разреженном разрезе}
    \section{Заключение}
    \bibliographystyle{plain}
    \bibliography{../bibtex/ir}
\end{document}
