\documentclass[12pt]{article}

\usepackage[utf8]{inputenc}
\usepackage[english, russian]{babel}
\usepackage{amssymb}
\usepackage{amsmath}
\usepackage{amsthm}
\usepackage{fullpage}
\usepackage{hyperref}
\usepackage{ccfonts}
\usepackage[T1]{fontenc}

\renewcommand{\bfdefault}{sbc}

\newcommand{\Pc}{\mathcal{P}}
\newcommand{\Rbb}{\mathbb{R}}

\newtheorem{theorem}{Теорема}
\newtheorem{lemma}{Лемма}

\begin{document}

    Пусть $G = (V, E, w)$~--- ненаправленный взвешенный граф.
    Обозначим $\Pc(i, j)$ множество вершин $G$, которые лежат на кратчайших путях между $i$ и $j$.

    Сформулируем задачу о минимальных метках в $G$ как квадратичную программу.
    Введем бинарные переменные $x_{i, k}$, которые равны единице тогда и только тогда, когда $k$ лежит в метке $i$.
    \begin{equation}
        \label{quad_prog}
        \begin{cases}
            x_{i,k} \in \{0, 1\} & \forall \; i, k \in V\\ 
            \sum_{k \in \Pc(i, j)} x_{i, k} x_{j, k} \geq 1 & \forall \; i, j \in V\\
            \sum_{i,k} x_{i,k} \to \min
        \end{cases}
    \end{equation}

    Рассмотрим $n^2$-мерную сферу в $(n^2+1)$-мерном пространстве $S^{n^2} \subset \Rbb^{n^2 + 1}$.
    Вместо бинарных переменных $x_{i,k}$ будем искать векторы $v_{i,k} \in S^{n^2}$.
    Кроме того, у нас будет выделенное направление $v^* \in S^{n^2}$.
    Рассмотрим следующую векторную программу.
    \begin{equation}
        \label{sdp}
        \begin{cases}
            v_{i,k}, v^* \in S^{n^2} & \forall \; i, k \in V \\
            \sum_{k \in \Pc(i, j)} \frac{1 + \langle v_{i,k}, v_{j, k} \rangle +
                                             \langle v_{i,k}, v^* \rangle +
                                             \langle v_{j,k}, v^* \rangle}{4} \geq 1 & \forall \; i, j \in V \\
            \sum_{i,k} \frac{1 + \langle v_{i,k}, v^* \rangle}{2} \to \min
        \end{cases}
    \end{equation}

    \begin{lemma}
        Векторная программа~(\ref{sdp}) является релаксацией квадратичной программы~(\ref{quad_prog}).
    \end{lemma}
    \begin{proof}
        Возьмем любое решение~(\ref{quad_prog}) $x_{i,k}$. Выберем $v^* \in S^{n^2}$ произвольно, а
        $v_{i,k}$ положим равным $v^*$, если $x_{i,k} = 1$, и $-v^*$ в противном случае.
        Далее непосредственно проверяется, что
        \begin{eqnarray*}
            x_{i,k} x_{j,k} &=& \frac{1 + \langle v_{i,k}, v_{j, k} \rangle +
                                             \langle v_{i,k}, v^* \rangle +
                                             \langle v_{j,k}, v^* \rangle}{4},\\
           x_{i,k} &=& \frac{1 + \langle v_{i,k}, v^* \rangle}{2}.
        \end{eqnarray*}
    \end{proof}

    Программа~(\ref{sdp}) решается за полиномиальное время с помощью SDP (ровно для этого мы и вынуждены искать
    векторы в пространстве размерности $n^2 + 1$).
    Обозначим $\mathrm{OPT}^*$ оптимальное значение целевой функции~(\ref{sdp}).

    Пусть для всех $i, j \in V$ размер $\Pc(i, j)$ не превосходит некоторого парамера $\Delta$.
    Мы покажем, как по решению~(\ref{sdp}) за полиномиальное время построить метки $G$ размера
    $O(\Delta^2) \cdot \mathrm{OPT}^*$, что дает преимущество перед алгоритмом Коэн, если $\Delta = o(\sqrt{\log n})$.

    \begin{theorem}
        По оптимальному решению~(\ref{sdp}) можно за полиномиальное время найти метки для $G$
        размера $O(\Delta^2) \cdot \mathrm{OPT}^*$.
    \end{theorem}
    \begin{proof}
        Обозначим $v$ оптимальное решение~(\ref{sdp}).
        Введем параметр $-1 < \eta \leq 1$, который мы зафиксируем позднее.
        Рассмотрим следующее решение~(\ref{quad_prog}):
        $$
            x_{i,k} = \begin{cases}
                0, & \langle v_{i,k}, v^* \rangle < \eta, \\
                1, & \langle v_{i,k}, v^* \rangle \geq \eta.
            \end{cases}
        $$
        Если мы выберем $\eta$ слишком большим, то $x$ не будет допустимым решением~(\ref{quad_prog}).
        Если же выбрать $\eta$ слишком маленьким, то значение целевой функции~(\ref{quad_prog}) будет слишком большим.
        Будем справляться с этими проблемами по очереди.

        \begin{lemma}
            \begin{equation}
                \label{obj_val}
                \sum_{i,k} x_{i,k} \leq \frac{2 \cdot \mathrm{OPT^*}}{1 + \eta}
            \end{equation}
        \end{lemma}
        \begin{proof}
            $$
                \mathrm{OPT}^* = \sum_{i,k} \frac{1 + \langle v_{i,k}, v^*\rangle}{2} \leq
                \sum_{i,k} x_{i,k} \cdot \frac{1 + \langle v_{i,k}, v^*\rangle}{2} \leq
                \sum_{i,k} x_{i,k} \cdot \frac{1 + \eta}{2}.
            $$
            Отсюда получаем~(\ref{obj_val}).
        \end{proof}

        Таким образом, нам достаточно доказать, что можно взять $\eta = -1 + \Omega(1 / \Delta^2)$ таким, что
        полученное $x$ будет допустимым решением~(\ref{quad_prog}).
        Для этого нам понадобится следующая геометрическая лемма.

        \begin{lemma}
            \label{triangle}
            Пусть $u, v, w \in \Rbb^N$ такие, что $\|u\| = \|v\| = \|w\| = 1$, и
            $\langle u, v \rangle + \langle v, w \rangle + \langle w, u \rangle \geq -1 + \varepsilon$,
            где $\varepsilon > 0$.
            Тогда $\langle u, v \rangle \geq -1 + \Omega(\varepsilon^2)$.
        \end{lemma}
        \begin{proof}
            Пусть $\langle u, v \rangle \leq -1 + \delta$, где $\delta > 0$.
            Тогда
            $$
                \langle v, w \rangle + \langle w, u \rangle = \langle u + v, w \rangle \leq \|u + v\| = 
                \sqrt{\|u\|^2 + \|v\|^2 + 2 \cdot \langle u, v \rangle} \leq \sqrt{2 \delta}.
            $$
            Мы придем к противоречию, если $-1 + \delta + \sqrt{2\delta} < -1 + \varepsilon$.
            Для этого можно взять $\delta = \Omega(\varepsilon^2)$.
        \end{proof}

        Теперь воспользуемся доказанной леммой для доказательства существования $\eta = -1 + \Omega(1 / \Delta^2)$
        такого, что $x$ является допустимым решением~(\ref{quad_prog}).

        Рассмотрим какие-то $i, j \in V$.
        Так как $v$ является решением~(\ref{sdp}), то имеем
        $$
            \sum_{k \in \Pc(i, j)} \frac{1 + \langle v_{i,k}, v_{j, k} \rangle +
                                             \langle v_{i,k}, v^* \rangle +
                                             \langle v_{j,k}, v^* \rangle}{4} \geq 1.
        $$
        Так как $|\Pc(i, j)| \leq \Delta$, то найдется $k \in \Pc(i, j)$ такое, что
        $$
            \frac{1 + \langle v_{i,k}, v_{j, k} \rangle +
                                             \langle v_{i,k}, v^* \rangle +
                                             \langle v_{j,k}, v^* \rangle}{4} \geq \frac{1}{\Delta}.
        $$
        Дальше применяем лемму~\ref{triangle} с $\varepsilon = 4 / \Delta$ и видим, что
        $$
            \langle v_{i,k}, v^* \rangle, \langle v_{j,k}, v^* \rangle \geq -1 + \Omega(1 / \Delta^2).
        $$
        Таким образом, теорема доказана.
    \end{proof}
\end{document}
